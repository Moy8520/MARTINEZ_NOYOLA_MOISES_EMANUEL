\documentclass[12pt,a4paper]{article}
\usepackage[utf8]{inputenc}
\usepackage{amsmath}
\usepackage{amsfonts}
\usepackage{amssymb}
\usepackage{makeidx}
\usepackage{graphicx}
\usepackage{lmodern}
\usepackage{kpfonts}
\usepackage{fourier}
\usepackage[left=2cm,right=2cm,top=2cm,bottom=2cm]{geometry}
\begin{document}
APLICACIONES DE PICK AND PLACE (RECOGER Y COLOCAR)\\
\\
 Luego de la introducción de la configuración de robot paralelo para desarrollar simuladores de vuelo, la aplicación más conocida y desarrollada de estos robot es en operaciones de pick and place. Es de destacar que el robot serial equivalente para este tipo de operaciones lo constituye el robot SCARA.\\
 \\
Las principales aplicaciones de los robot antes descritos se encuentran en Industrias de empaquetado como la de alimentos, para el manejo de células fotovoltaicas, manejo de instrumentos médicos, corte laser a alta velocidad, así como también mecanizado de piezas de madera. También han sido empleados para la manejo de verduras, para mayor detalle se refiere al lector a la referencia.\\
\\
APLICACIONES EN CENTRO DE MECANIZADO. \\
\\
Otra de las aplicaciones prácticas de los robots paralelos se encuentra en el área de desarrollo de centros de mecanizado. En este campo es común que los desarrolladores se refieran al robot paralelo como Mecanismo Cinemático Paralelo.\\
\\
APLICACIONES EN LA CIRUGÍA ROBÓTICA\\
\\
 La cirugía mínimamente invasiva representa una de las áreas donde la introducción de robot produce un gran impacto, sobre todo mejorando las prestaciones de la cirugía laparoscópica, ya que aumenta la habilidad del cirujano a la hora de realizar una operación (mayor precisión, evita el movimiento errático del pulso de la mano). Con la cirugía robótica se han logrado avances como realizar una operación mediante orificios de 10 mm en el cuerpo del paciente. \\
 \\
 APLICACIONES EN ROBÓTICA PARA REHABILITACIÓN.\\
 \\
 La rehabilitación robótica se presenta como otro de los campos de mayor interés en la actualidad, sirviendo de asistencia al trabajo arduo de los fisioterapeutas, además de que logra una mejor coordinación para los ejercicios de rehabilitación y mayor precisión en el diagnóstico de lesiones y la medición de la evolución de los pacientes. La rehabilitación y diagnosis de las extremidades inferiores es muy frecuente debido a la gran cantidad de accidentes a los que están expuestas estas extremidades, de hecho, en el campo de los deportes suelen presentarse muy a menudo.\\
 \\
 Los robots paralelos presentan su mayor aplicación en el desarrollo de robot para rehabilitación de tobillo. En la última década se han desarrollado importantes trabajos en el área de la robótica para la rehabilitación y entre los que se han desarrollado prototipos de dispositivos de rehabilitación para la articulación de tobillo.




\end{document}
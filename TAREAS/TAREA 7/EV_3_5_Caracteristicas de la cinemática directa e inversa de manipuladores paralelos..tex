\documentclass[12pt,a4paper]{article}
\usepackage[utf8]{inputenc}
\usepackage{amsmath}
\usepackage{amsfonts}
\usepackage{amssymb}
\usepackage{makeidx}
\usepackage{graphicx}
\usepackage{lmodern}
\usepackage{kpfonts}
\usepackage{fourier}
\usepackage[left=2cm,right=2cm,top=2cm,bottom=2cm]{geometry}
\begin{document}
Para que un robot manipulador pueda realizar una tarea específica, primero se realiza el análisis cinemático de posición, donde se debe de conocer la localización del efector final con respeto al sistema inercial. En el análisis cinemático de posición hay dos tipos, la cinemática directa y la cinemática inversa.\\
\\
Cinemática directa \\
\\
Se denomina cinemática directa a una técnica usada en gráficos 3D por computadora, para calcular la posición de partes de una estructura articulada a partir de sus componentes fijas y las transformaciones inducidas por las articulaciones de la estructura.\\
\\
La cinemática directa se refiere al uso de ecuaciones cinemáticas para calcular la posición de su actuador final a partir de valores específicos denominado parámetros. Las ecuaciones cinemáticas de un robot son usadas en robots, juegos de computadoras y la animación.\\
\\
Métodos para el análisis de la cinemática directa.\\
\\
1.	Transformación de matrices.\\
2.	Geometría.\\
3.	Transformación de coordenadas.\\
\\
El problema cinemático directo consiste en determinar cuál es la posición y orientación del extremo final del robot, con respecto a un sistema de coordenadas que se toma como referencia, conocidos los valores de las articulaciones y los parámetros geométricos de los elementos del robot.\\
\\
Cinemática inversa\\
\\
 La cinemática inversa, que consiste en calcular las transformaciones necesarias en las articulaciones de una estructura, de modo que su extremo se coloque en una posición determinada.\\
 \\
 Es la técnica que permite determinar el movimiento de una cadena de articulaciones para lograr que un actuador final se ubique en una posición concreta. El cálculo de la cinemática inversa es un problema complejo que consiste en la resolución de una serie de ecuaciones cuya solución normalmente no es única.\\

El objetivo de la cinemática inversa es encontrar los valores que deben tomar las coordenadas articulares del robot para que su extremo se posicione y oriente según una determinada localización espacial. \\
\\
Siempre que se especifica una posición de destino y una orientación en términos cartesianos, debe calcularse la cinemática inversa del dispositivo para poder despejar los ángulos de articulación requeridos. Los sistemas que permiten describir destinos términos cartesianos son capaces de mover el manipulador a puntos que nunca fueron capaces de mover el espacio de trabajo a los cuales tal vez nunca haya ido antes.


\end{document}
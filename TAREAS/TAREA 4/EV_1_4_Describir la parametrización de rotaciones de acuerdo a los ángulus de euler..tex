\documentclass[12pt,a4paper]{article}
\usepackage[utf8]{inputenc}
\usepackage{amsmath}
\usepackage{amsfonts}
\usepackage{amssymb}
\usepackage{makeidx}
\usepackage{graphicx}
\usepackage{lmodern}
\usepackage{kpfonts}
\usepackage{fourier}
\usepackage[left=2cm,right=2cm,top=2cm,bottom=2cm]{geometry}
\begin{document}
Parametrización de rotaciones deacuerdo a los angulos de Euler.\\
\\
•El movimiento general de un sólido puede descomponerse en una traslación de un punto seguida de una rotación del sólido alrededor de este punto (es decir, tomándolo como fijo en la rotación). En la parametrización del movimiento de un sólido, los grados de se descomponen en traslación y rotación.\\
Para la rotación, existen diferentes formas de parametrizarla, cada una con sus ventajas e inconvenientes. Así, tenemos:\\
\\
•	Dar directamente la matriz de rotación. Esta matriz tiene 9 elementos sometidos a seis vínculos, lo cual multiplica el número de ecuaciones necesarias para la descripción del movimiento.\\
Puede demostrarse que cualquier rotación de un sólido puede expresarse como la composición de tres rotaciones elementales alrededor de ejes diferentes. A su vez, estas rotaciones pueden considerarse en torno a unos ejes fijos o en torno a unos ejes intrínsecos.\\
\\
Aquí consideraremos especificaremos una composición de rotaciones que nos llevarán desde un sistema exterior considerado fijo (“sólido 1”) hasta un sistema ligado al sólido (“sólido 4”) mediante sólidos intermedios. Para ello:\\
\\
•	Primero efectuamos una rotación de un ángulo φ en torno a un eje del sólido 1 que nos lleva a un sólido intermedio “2”.\\
\\
•	A continuación rotamos un ángulo θ en torno a un eje del sólido 2, que nos lleva a un sólido intermedio “3”.\\
\\
•	Por último, giramos un cierto ángulo ψ en torno a un eje del sólido 3, lo que nos lleva hasta el sólido móvil 4.\\
\\
La elección de qué ejes son los de rotación puede hacerse de diferentes formas. La elección que haremos aquí, que es la más habitual en el uso de los ángulos de Euler consiste en la secuencia Z1 − X2 − Z3. El que se repita uno de los ejes (aunque no sean coincidentes, por la rotación intermedia) es lo que define a esta descomposición como de ángulos de Euler. Si fueran todos diferentes (Z1 − Y2 − X3, por ejemplo) serían de Tait-Bryan, que veremos después.\\
Analizaremos la expresión de la posición, velocidad y aceleración en función de estos ángulos y sus derivadas temporales.\\
\\
Referencias\\
@inproceedings{bergamini2015parametrizaciones,
  title={Parametrizaciones de Movimientos R{\'\i}gidos en 3D para Visi{\'o}n Artificial},
  author={Bergamini, Mar{\'\i}a Lorena and Kamlofsky, Jorge Alejandro},
  booktitle={Actas del 3er. Congreso Nacional de Ingenier{\'\i}a Inform{\'a}tica/Sistemas de Informaci{\'o}n CoNaIISI},
  pages={365--369},
  year={2015}
}


\end{document}
\documentclass[12pt,a4paper]{article}
\usepackage[utf8]{inputenc}
\usepackage{amsmath}
\usepackage{amsfonts}
\usepackage{amssymb}
\usepackage{makeidx}
\usepackage{graphicx}
\usepackage{lmodern}
\usepackage{kpfonts}
\usepackage{fourier}
\usepackage[left=2cm,right=2cm,top=2cm,bottom=2cm]{geometry}
\begin{document}
$$Método geométrico$$\\

Este método es útil para robots con pocos grados de libertad.\\
El procedimiento se basa en encontrar un número suficiente de relaciones geométricas en las que intervendrán las coordenadas del extremo del robot, sus coordenadas articulares y las dimensiones físicas de sus elementos.\\
Con este método determinamos posiciones por medio de resolución de triángulos.\\
\\
$$\includegraphics[scale=1]{../../geo 2.PNG} $$
\\
$$\includegraphics[scale=1]{../../geo 3.PNG} $$
$$\includegraphics[scale=1]{../../geometrico 1.PNG} $$
\\
$$Método  algebraico.$$\\
 Como se discutió en la sección anterior sólo en casos especiales pueden los robots manipuladores resolverse de manera cerrada y analítica. Para ello el manipulador debe cumplir alguna de las siguientes condiciones: \\
• Tres ejes de articulación adyacentes intersectantes en un punto y muchos αi iguales a 0 o ±90 grados.\\
 • Tres ejes de articulación adyacentes son paralelos entre sí.\\
Las siguientes equivalencias trigonométricas son de gran utilidad para simplificar las expresiones matriciales a partir de las condiciones anteriores:\\
 sen(a+b)=sen(a)cos(b)+cos(a)sen(b)\\
cos(a+b)=cos(a)cos(b) sen(a)sen(b) − sen(a + b + c) = sen(a)cos(b)cos(c)+ cos(a)sen(b)cos(c) − sen(a)sen(b)sen(c)+ cos(a)cos(b)sen(c) \\
cos(a + b +c) = cos(a)cos(b)cos(c) sen(a)sen(b)cos(c) sen(a)cos(b)sen(c) cos(a)sen(b)sen(c) \\
\\
$$Desacoplo cinematico.$$\\
El método de desacoplo cinemático saca partido entre: Posición y orientación. Para ello, dada una posición y orientación final deseadas, establece las coordenadas del punto de corte de los 3 últimos ejes (muñeca del robot) calculándose los valores de las tres primeras variables articulares (q1, q2, q3) que consiguen posicionar este punto.\\
\\
Paso 1: Encontrar q1, q2, q3 de tal manera que la muñeca de centro oc tiene coordenadas dadas por:\\
$$\includegraphics[scale=1]{../../des 1.PNG} $$
Paso 2: Usando las variables determinadas en el paso 1, evaluar 0 R3.\\
Paso 3: Buscar un conjunto de ángulos de Euler correspondientes a la matriz de rotación.\\
$$\includegraphics[scale=1]{../../des 2.PNG} $$
\\
Se utilizaran los vectores Pm y Pr, que van desde el origen del sistema asociado a la base del robot ( S0) hasta los puntos centro de la muñeca y fin del robot, respectivamente.\\


\end{document}
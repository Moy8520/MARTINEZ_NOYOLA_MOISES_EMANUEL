\documentclass[12pt,a4paper]{article}
\usepackage[utf8]{inputenc}
\usepackage{amsmath}
\usepackage{amsfonts}
\usepackage{amssymb}
\usepackage{makeidx}
\usepackage{graphicx}
\usepackage{lmodern}
\usepackage{kpfonts}
\usepackage{fourier}
\usepackage{chicago}
\begin{document}
\section{ROTACIONES Y CUATERNIOS}
Los Cuaternios son una extensión de los números reales, similar a la de los números complejos son una extensión generada de manera análoga añadiendo las unidades imaginarias: i, j y k a los números reales.
$$\includegraphics[scale=1]{1.PNG} $$ 
El producto se realiza componente a componente, y está dado en su forma completa por:
$$\includegraphics[scale=1]{2.PNG}  $$
 Es posible escribir el cociente de dos cuaternios como:
 $$\includegraphics[scale=1]{3.PNG}$$ 
 La exponenciación cuaternios está relacionada con funciones trigonométricas:
 $$\includegraphics[scale=1]{4.PNG} $$
 Son útiles para representar rotaciones de un objeto respecto a otro.
 $$\includegraphics[scale=1]{5.PNG} $$
 $$Rotaciones$$
 Al rotar un vector (x, y) ∈ R 2 alrededor del origen por un angulo θ obtenemos otro vector (x 0 , y0 ) ∈ R 2 cuyas coordenadas son:
 $$\includegraphics[scale=1]{6.PNG} $$
\bibliographystyle{chicago}
\bibliography{EV_1_3_Par de rotacion y cuaternios}

\end{document}
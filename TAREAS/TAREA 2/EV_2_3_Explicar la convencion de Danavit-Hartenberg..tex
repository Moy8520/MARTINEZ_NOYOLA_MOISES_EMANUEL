\documentclass[12pt,a4paper]{article}
\usepackage[utf8]{inputenc}
\usepackage{amsmath}
\usepackage{amsfonts}
\usepackage{amssymb}
\usepackage{makeidx}
\usepackage{graphicx}
\usepackage{lmodern}
\usepackage{kpfonts}
\usepackage{fourier}
\usepackage[left=2cm,right=2cm,top=2cm,bottom=2cm]{geometry}
\title{Convención Danavit-Hartenberg}
\begin{document}
\section{\textbf{Convención Danavit-Hartenberg}}\\
Se trata de un procedimieto sistemático para describir la estructura cinemática de una cadena articulada constituida por articulaciones con. un solo grado de libertad.\\
1.	Tenemos que enumerar los n+1 eslabones de 0 a 1, comenzando desde la base y terminando en el efector final.\\
2.	Identificar los ejes de cada articulación. Si es rotacional será el eje de giro, y si es prismática será el eje a lo largo del cual se produce el desplazamiento.\\
3.	Enumerar los ejes de 1 a n comenzando desde el que une eslabón base con el eslabón 1.\\
$$\includegraphics[scale=1]{../../1.PNG} \\$$
4.	 Para i de 0 a n-1 : situar el eje en el eje Zi de articulación i+1 .\\
5.	El eje Zn se colocará en el extremo del último eslabón, en la misma dirección que el Zn-1.\\
6.	Situar el origen del sistema de la base {So} en cualquier punto del eje Zo.\\
7.	Para  i de 1 a n: situar el eje a partir del punto donde se definió el {Si } sobre la recta que es perpendicular simultáneamente al eje Zi  y al eje Zi-1. Si los ejes Zi y Zi-1  se cortan el eje X debe ser perpendicular a ambos. El sentido es indiferente.\\
8.	El  Xo se puede colocar libremente. Puede resultar útil que esté alineado con el X1.\\
9.	Para  i de 0 a n: colocar el eje Yi de modo que forme un sistema dextrógiro con los ejes Xi y Zi .\\
$$\includegraphics[scale=1]{../../2.PNG} $$

Bibliografia \\
@article{balasubramanian2011denavit,
  title={The Denavit Hartenberg Convention},
  author={Balasubramanian, Ravi},
  journal={USA: Robotics Insitute Carnegie Mellon University},
  year={2011}
}





\end{document}